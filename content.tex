% content of the paper
\section{Введение}
% В этой статье мы расскажем про устройство глагольных полей 'искать' и 'находить' в языке кунбарланг такой-то семьи. Это язык, на котором говорят там-то и тот-то. Может быть, карта. Еще какая-нибудь социолингвистика -- про угрожаемый статус, про то, что все носители билингвы и сколько их. Может быть, яркие типологические черты.
В этой статье мы расскажем про устройство глагольных полей `искать' и `находить' (а также про некоторые смежные сюжеты) в не-пама-ньюнганском языке кунбарланг (гунвингская семья, северная Австралия). На сегодняшний день на кунбарланге говорят приблизительно 35 человек в трёх населённых пунктах в Арнем-ленде на крайнем севере Австралии. Все взрослые кунбарланги многоязычны; большинство свободно говорят на не-пама-ньюнганских языках маунг (семья ивайджа) или нджеббана (манингридская семья), а также в разной степени владеют английским языком. Этнические дети-кунбарланги не усваивают родной язык вследствие конкуренции с этими другими языками, и кунбарланг находится под угрозой исчезновения.

С точки зрения структуры, \кун\ является языком полисинтетического строя с такими характерными чертами как полиперсональное согласование, инкорпорация имён и наречий в глагольную словоформу, грамматически свободный порядок слов и про-дроп. Все глагольные формы \кун а финитны, и любая такая форма может быть полноценной самостоятельной клаузой. Именная морфология слабо развита, структурный падеж сохраняется только у личных местоимений, на существительных изредка маркируются пространственные падежи. Таким образом, в принципе возможна неоднозначность в интерпретации грамматической функции произвольной именной группы; на практике она возникает редко в силу семантико-прагматических соображений. %@ Основной согласовательной категорией в именной группе является грамматический род.
В \кун е четыре грамматических рода\footnote{Если учитывать указательные местоимения, то можно выделять пять грамматических родов.}, ну и хули?

Кунбарланг мало описан: существует короткая тагмемная грамматика \cite{harris69} и ряд неопубликованных работ К. Колман --- дипломная работа \cite{coleman82}, рукописи разговорника \parencite{wordgra} и словаря \parencite{coleman10}. В настоящий момент в университете Мельбурна готовится полное грамматическое описание языка \parencite{ikwlg}. 
Материал для настоящей статьи был собран вторым автором во время полевой работы в деревне Варруви (о.\ Южный Голбурн). %@ add info about the exx from others' fieldwork


%Что касается устройства лексических полей, здесь кунбарланг проявляет типичную для австралийских языков странность: неожиданные для носителей европейских языков поля оказываются бедными и наоборот. Например, родство и всякие животные/природа -- богатые. А еда и что-то еще бедные. Глагольные поля -- это скорее последний случай. Много доминантных систем, глаголов с очень широким значением (например). Это не значит, что периферийных нет. Аналогия с английским.

\section{Искать}
Большинство ситуаций искания описываются в \кун е покрывается доминантным глаголом -\textit{yawanj} `\yaw'. Так ищут как одушевленный \rex{human}, так и неодушевленный \rex{nonhuman} объекты.\footnote{В статье используются следующие сокращения: \printglossary[style=inline,type=\leipzigtype]}

\ex<human>\begingl
\gla \textbf{Nga}-\textbf{yawanj} nakarrmanj nga-mabulunj nganjdji-wokdja.//
\glb \Fsg.\Real-\yaw.\Np{} племянник \nga-хотеть.\Np{} \Fdu.\Excl.\Fut-говорить.\Np{}//
\glft `Я ищу своего племянника, я хочу с ним поговорить.'//%gDoc:g2
\endgl \xe

\ex<nonhuman> \begingl
\gla \textbf{Nga}-\textbf{yawanj} bi-ngaybu sunglass nga-warre \textbf{nga}-\textbf{yawanj}.//
\glb \nga-\yaw.\Np{} \bi-я.\Obl{} sunglass \nga-\warre.\Np{} \nga-\yaw.\Np{}//
\glft `Я ищу свои солнечные очки.'//%gDoc:g3
\endgl \xe

Объект поиска может быть не только референтным, как в примерах (\getref{human}--\getref{nonhuman}), но и нереферентным. Например, с помощью этого глагола можно искать себе друзей \rex{friend} или пресную воду \rex{water}.

\ex<friend> \begingl
\gla Ninda nawalak na-kerrkung \textbf{ka}-\textbf{bun}-\textbf{yawanj} \textbf{na}-\textbf{barrkidbe} \textbf{djarrangalanj} nayi bi-rnungu friend.//
\glb \Dem.\Prox.\Cli{} ребёнок \Cli-новый \Tsg.\Real-\Tsg.\Obj-\yaw.\Np{} \Cli-другой мальчик \Nm.\Cli{} \bi-он.\Obl{} друг//
\glft `Это ребенок в школе новенький, он ищет друга.'//%gDoc:g10; в доке --- nakerrbung
\endgl \xe

\ex<water>\begingl
\gla blabla storks are looking for water//
\glb blabla//
\glft `blabla'//
\endgl \xe

Морфологически на глаголе маркируются лицо и число субъекта и объекта поиска. Объект третьего лица единственного числа обычно нулевой и выражается префиксом \textit{bun}- только если его референт одушевленный, а субъект --- тоже третьего лица единственного числа \rex{friend}. 

%Во всех остальных случаях (которых в наших примерах большинство) этот объект нулевой (как, например, в \rex{human}). 

Третий участник ситуации поиска --- место --- на глаголе не маркируется. Он выражается отдельной предложной группой с универсальным предлогом \textit{korro} `\korro':
\ex<ex:pp>\begingl
\gla example.//
\glb //
\glft `transl.'\trailingcitation{[src]}//
\endgl\xe

В этом отношении глагол \textit{yawanj} в \кун е стоит в одном ряду в русским \textit{искать} и английским \textit{seek}. Во всех этих словах место поиска не входит в число аргументов глагола и может быть выражено только адъюнктом. Согласно классификации когнитивных профилей Ван Хенке (2003), все эти глаголы относятся к типу \textsc{seek}: выделенный участник ситуации в таких словах -- желаемый объект, а область поиска занимает периферийную позицию. В \кун е есть два способа сместить акцент с объекта на место. 

Первый способ --- глагол \textit{birrdjuwa} (этимологически от \textit{birr}- `рука' и \textit{djuwa} `протыкать'). Этот глагол не относится к полю поиска и в прямом значении описывает уборку метлой или граблями \rex{rake}:

\ex<rake>\begingl
\gla example.//
\glb //
\glft `transl.'\trailingcitation{[src]}//
\endgl\xe

Характерное мультипликативное движение рукой в процессе такой уборки послужило основой метафорических переносов: `расчесывать волосы', `чистить зубы', 'открывать дверь' и `перелистывать страницы'.

\ex<combhair>\begingl
\gla example.//
\glb //
\glft `transl.'\trailingcitation{[src]}//
\endgl\xe

\ex<flip>\begingl
\gla example.//
\glb //
\glft `transl.'\trailingcitation{[src]}//
\endgl\xe

Следующим семантическим переносом акцент смещается с действия на его цель. Здесь и возникает идея поиска. Так, расчищать граблями землю можно с целью  найти что-либо под ворохом листьев или веток. А перелистывать страницы книги можно для того, чтобы найти какую-то конкретную информацию.

\ex<searchground>\begingl
\gla example.//
\glb //
\glft `transl.'\trailingcitation{[src]}//
\endgl\xe

\ex<searchbook>\begingl
\gla example.//
\glb //
\glft `transl.'\trailingcitation{[src]}//
\endgl\xe

%прямой объект -- место. а объект поиска здесь вообще нельзя выразить. если очень надо -- тогда снова первое искать. 
%нетривиальная прямая семантика глагола сильно ограничивает возможности референта. так, земля и книга могут быть, но этим варианты практически исчерпываются. например, когда носителю предложили предложение с человеком, он проинтерпретировал.


%Также на глаголе может быть выражен бенефициант. 
%Есть два способа сделать акцент на месте. Глагол `тянуть' и конструкция с инкорпорацией.

%\ex<ex:wfound>\begingl
%\gla Nga-warrenj nga-yawang la babi la \textbf{nga}-\textbf{warrenj} \textbf{la} \textbf{nga}-\textbf{rnay}.//
%\glb \Fsg.\Real-\warre.\Pst{} \Fsg.\Real-искать.\Pst{} \la{} позже \la{} \Fsg.\Real-\warre.\Pst{} \la{} \Fsg.\Real-видеть.\Pst{}//
%\glft `Я искал нечто, а потом (\textbf{я шарился и}) \textbf{нашёл}.'\trailingcitation{[IK1-170610\_1SY-02/54:02--10]}//
%\endgl\xe

\section{Находить}
% А поле \textsc{находить} покрывается ещё более доминантным глаголом.

\section{Заключение}

%%% Local Variables:
%%% mode: latex
%%% TeX-master: "main"
%%% End:
