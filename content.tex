% content of the paper
\section{Введение}
% В этой статье мы расскажем про устройство глагольных полей 'искать' и 'находить' в языке кунбарланг такой-то семьи. Это язык, на котором говорят там-то и тот-то. Может быть, карта. Еще какая-нибудь социолингвистика -- про угрожаемый статус, про то, что все носители билингвы и сколько их. Может быть, яркие типологические черты.
В этой статье мы расскажем про устройство глагольных полей `искать' и `находить' (а также про некоторые смежные сюжеты) в не-пама-ньюнганском языке кунбарланг (гунвингская семья, северная Австралия). На сегодняшний день на кунбарланге говорят приблизительно 35 человек в трёх населённых пунктах в Арнем-ленде на крайнем севере Австралии. Все взрослые кунбарланги многоязычны; большинство свободно говорят на не-пама-ньюнганских языках маунг (семья ивайджа) или нджеббана (манингридская семья), а также в разной степени владеют английским языком. Этнические дети-кунбарланги не усваивают родной язык вследствие конкуренции с этими другими языками, и кунбарланг находится под угрозой исчезновения.

С точки зрения структуры, \кун\ является языком полисинтетического строя с такими характерными чертами как полиперсональное согласование, инкорпорация имён и наречий в глагольную словоформу, грамматически свободный порядок слов и про-дроп. Все глагольные формы \кун а финитны, и любая такая форма может быть полноценной самостоятельной клаузой. Именная морфология слабо развита, структурный падеж сохраняется только у личных местоимений, на существительных изредка маркируются пространственные падежи. Таким образом, в принципе возможна неоднозначность в интерпретации грамматической функции произвольной именной группы; на практике она возникает редко в силу семантико-прагматических соображений. %@ Основной согласовательной категорией в именной группе является грамматический род.
В \кун е четыре грамматических рода\footnote{Если учитывать указательные местоимения, то можно выделять пять грамматических родов.}, ну и хули? %@

Кунбарланг мало описан: существует короткая тагмемная грамматика \cite{harris69} и ряд неопубликованных работ К. Колман --- дипломная работа \cite{coleman82}, рукописи разговорника \parencite{wordgra} и словаря \parencite{coleman10}. В настоящий момент в университете Мельбурна готовится полное грамматическое описание языка \parencite{ikwlg}. 
Материал для настоящей статьи был собран вторым автором во время полевой работы в деревне Варруви (о.\ Южный Голбурн). %@ add info about the exx from others' fieldwork

%@ написать, что в кбл нет страдательного залога

В лексической семантике \кун\ демонстрирует характерные для австралийских языков черты. В связи со сложной системой социальной организации (классификаторная система родства), %@ как это по-русски-то??
термины родства --- как термины обращения, так и термины референциальные --- %@ а это?? + ПРИМЕР?
очень богаты и разнообразны. Также очень подробен и богат этнобиологический лексикон. Вследствие того, что австралийские аборигены до колонизации жили охотой и собирательством, и такой образ жизни требовал тонкого понимания живой природы, в любом австралийском языке есть большой пласт лексики, посвящённой биологическим видам. Тесная взаимосвязь с природой проявляется также и в системах классификации флоры и фауны в языке (в \кун е это грамматические роды) и, вне языка, в таких культурных практиках как танец, песня, изобразительное искусство.

В то же время, некоторые другие области лексики развиты непривычно слабо для носителя, например, индоевропейского языка. Так, например, в \кун е при изобилии видовых терминов часто отсутствуют родоввые термины. Много названий для отдельных видов ракообразных и моллюскоы, но нет слова для ракообразных вообще или для моллюсков вообще. Одним словом \textit{maworord} обозначаются лист, цветок, лепесток и травинка. Практически полностью отсутствуют кулинарные термины: есть слова \textit{neyang} `растительная еда' и \textit{kakkin} `мясо', но нет никаких различий по способу приготовления пищи; обычно просто называется вид съедаемого растения или животного.

%Что касается устройства лексических полей, здесь кунбарланг проявляет типичную для австралийских языков странность: неожиданные для носителей европейских языков поля оказываются бедными и наоборот. Например, родство и всякие животные/природа -- богатые. А еда и что-то еще бедные. Глагольные поля -- это скорее последний случай. Много доминантных систем, глаголов с очень широким значением (например). Это не значит, что периферийных нет. Аналогия с английским.

\section{Искать}
Большинство ситуаций искания описываются в \кун е покрывается доминантным глаголом -\textit{yawanj} `\yaw'. Так ищут как одушевленный \rex{human}, так и неодушевленный \rex{nonhuman} объекты.\footnote{В статье используются следующие сокращения: \printglossary[style=inline,type=\leipzigtype]}

\ex<human>\begingl
\gla \textbf{Nga}-\textbf{yawanj} nakarrmanj nga-mabulunj nganjdji-wokdja.//
\glb \Fsg.\Real-\yaw.\Np{} племянник \nga-хотеть.\Np{} \Fdu.\Excl.\Fut-говорить.\Np{}//
\glft `Я ищу своего племянника, я хочу с ним поговорить.'//%gDoc:g2
\endgl \xe

\ex<nonhuman> \begingl
\gla \textbf{Nga}-\textbf{yawanj} bi-ngaybu sunglass nga-warre \textbf{nga}-\textbf{yawanj}.//
\glb \nga-\yaw.\Np{} \bi-я.\Obl{} sunglass \nga-\warre.\Np{} \nga-\yaw.\Np{}//
\glft `Я ищу свои солнечные очки.'//%gDoc:g3
\endgl \xe

Объект поиска может быть не только референтным, как в примерах (\getref{human}--\getref{nonhuman}), но и нереферентным. Например, с помощью этого глагола можно искать себе друзей \rex{friend} или пресную воду \rex{water}.

\ex<friend> \begingl
\gla Ninda nawalak na-kerrkung \textbf{ka}-\textbf{bun}-\textbf{yawanj} \textbf{na}-\textbf{barrkidbe} \textbf{djarrangalanj} nayi bi-rnungu friend.//
\glb \Dem.\Prox.\Cli{} ребёнок \Cli-новый \Tsg.\Real-\Tsg.\Obj-\yaw.\Np{} \Cli-другой мальчик \Nm.\Cli{} \bi-он.\Obl{} друг//
\glft `Это ребенок в школе новенький, он ищет друга.'//%gDoc:g10; в доке --- nakerrbung
\endgl \xe

\ex<water>\begingl
\gla Ki-warreni ki-wokdji \textbf{ki}-\textbf{yawani} \textbf{njunjuk} bonj\char`~bonj.//
\glb \Tsg.\irrpst-\warre.\irrpst{} \Tsg.\irrpst-говорить.\irrpst{} \Tsg.\irrpst-\yaw.\irrpst{} вода \rdp\bonj{}//
\glft `Они [журавли], бывало, летят, кричат, снова ищут воду.'\trailingcitation{[20150212AS02\_brolga\_transcript/01:59--02:03]}//
\endgl \xe

Морфологически на глаголе маркируются лицо и число субъекта и объекта поиска. Объект третьего лица единственного числа обычно нулевой и выражается префиксом \textit{bun}- только если его референт одушевленный, а субъект --- тоже третьего лица единственного числа \rex{friend}. 

%Во всех остальных случаях (которых в наших примерах большинство) этот объект нулевой (как, например, в \rex{human}). 

Третий участник ситуации поиска --- место --- на глаголе не маркируется. Он выражается отдельной предложной группой с универсальным предлогом \textit{korro} `\korro':
%@ need to listen to this example...
\ex<ex:pp>\begingl
\gla Nganj-ka nganj-yawanj nayi kunj \textbf{korro} kadda-dja kadda-bardi-djinj njunjuk korro kungad.//
\glb \Fsg.\Fut-идти.\Np{} \Fsg.\Fut-\yaw.\Np{} \Nm.\Cli{} кенгуру \korro{} \Tpl.\Real-стоять.\Np{} \Tpl.\Real-жидкость-есть.\Np{} вода \korro{} заводь//
\glft `Я пойду искать кенгуру там, где они пьют воду, у заводи.'//%mm_160817
\endgl\xe

В этом отношении глагол \textit{yawanj} в \кун е стоит в одном ряду в русским \textit{искать} и английским \textit{seek}. Во всех этих словах место поиска не входит в число аргументов глагола и может быть выражено только адъюнктом. В \кун е есть два способа передать идею поиска с акцентом на месте. 
%Согласно классификации когнитивных профилей Ван Хенке (2003), все эти глаголы относятся к типу \textsc{seek}: выделенный участник ситуации в таких словах -- желаемый объект, а область поиска занимает периферийную позицию. 

Первый способ --- глагол \textit{birrdjuwa} (этимологически от \textit{birr}- `рука' и \textit{djuwa} `протыкать'). В прямом значении этот глагол описывает уборку метлой или граблями:

\ex<rake>\begingl
\gla example.//
\glb //
\glft `transl.'\trailingcitation{[src]}//
\endgl\xe

Образ очищения поверхности от загораживающих её элементов послужил основой ряду метафорических переносов. Например, этим глаголом небо может
расчиститься от облаков \rex{sky}, им же можно открыть дверь (буквально, очистить проход от двери, \rex{door}) или перелистывать страницы (очищать новые страницы от загораживающих их старых).

\ex<sky>\begingl
\gla example.//
\glb //
\glft `transl.'\trailingcitation{[src]}//
\endgl\xe

\ex<door>\begingl
\gla example.//
\glb //
\glft `transl.'\trailingcitation{[src]}//
\endgl\xe

Расчищать пространство можно не только с целью сделать его чистым, но и для того чтобы что-нибудь найти. Здесь и возникает идея поиска. Так, расчищать граблями землю можно с целью  найти что-либо под ворохом листьев или веток \rex{searchground}. А перелистывать страницы книги можно для того, чтобы найти в книге какую-то конкретную информацию \rex{searchbook}. В таких употреблениях \textit{birrdjuwa} сближается с русскими глаголами \textit{обыскивать} и \textit{рыться}.

\ex<searchground>\begingl
\gla example.//
\glb //
\glft `transl.'\trailingcitation{[src]}//
\endgl\xe

\ex<searchbook>\begingl
\gla example.//
\glb //
\glft `transl.'\trailingcitation{[src]}//
\endgl\xe

Место поиска (что именно очищается) у \textit{birrdjuwa} --- прямой объект. Обычно этот аргумент выражается независимой именной группой, но иногда дублируется в виде инкорпорированного корня в глаголе \rex{searchground}. Объект поиска при этом предикате выразить нельзя. Для того, чтобы его назвать, необходимо употребить доминантную лексему поля\textit{yawanj} \rex{searchbook}.

Нетривиальная семантика глагола \textit{birrdjuwa} в прямых употреблениях сильно ограничивает круг допустимых ситуаций поиска. Для того, чтобы употребление глагола было возможно, ситуация должна предполагать физическое очищение пространства от ненужных объектов. Так, искать что-то в груде мусора, в земле или в 
книге при помощи этого глагола можно а, например, искать себе друзей (как бы очищать общество от ненужных людей в поиске друга) нельзя. Когда носителю показали такое предложение (`Он \textit{birrdjuwa} детей') и спросили, что оно могло бы значить, он проинтерпретировал его следующим образом: "Он стаскивает с детей одеяла" \rex{children}.

\ex<children>\begingl
\gla example.//
\glb //
\glft `transl.'\trailingcitation{[src]}//
\endgl\xe

Другой способ выразить идею поиска с акцентом на месте --- глагол \textit{rlakka}, который в прямом значении передает смысл 'бросать' \rex{throw}. Если употребить этот глагол в редуплицированной форме \textit{rla-rlakka} с существительным 'сумка' в качестве прямого объекта, фраза приобрает значение 'рыться в сумке' \rex{bag}. Редупликация здесь передает идею множественности действия ('разбрасывать'). 

\ex<throw>\begingl
\gla example.//
\glb //
\glft `transl.'\trailingcitation{[src]}//
\endgl\xe

\ex<bag>\begingl
\gla example.//
\glb //
\glft `transl.'\trailingcitation{[src]}//
\endgl\xe

Так же, как и лексема \textit{birrdjuwa}, \textit{rla-rlakka} допускает дублирование прямого объекта в виде инкорпорированного корня \rex{baginc}. С этим глаголом тоже нельзя выразить искомый объект --- ни аргументом, ни адъюнктом --- для этого употребляется доминантное слово \textit{yawanj}.

\ex<baginc>\begingl
\gla example.//
\glb //
\glft `transl.'\trailingcitation{[src]}//
\endgl\xe

%\ex<ex:wfound>\begingl
%\gla Nga-warrenj nga-yawang la babi la \textbf{nga}-\textbf{warrenj} \textbf{la} \textbf{nga}-\textbf{rnay}.//
%\glb \Fsg.\Real-\warre.\Pst{} \Fsg.\Real-искать.\Pst{} \la{} позже \la{} \Fsg.\Real-\warre.\Pst{} \la{} \Fsg.\Real-видеть.\Pst{}//
%\glft `Я искал нечто, а потом (\textbf{я шарился и}) \textbf{нашёл}.'\trailingcitation{[IK1-170610\_1SY-02/54:02--10]}//
%\endgl\xe

\section{Находить}
% А поле \textsc{находить} покрывается ещё более доминантным глаголом.

\section{Заключение}

%%% Local Variables:
%%% mode: latex
%%% TeX-master: "main"
%%% End:
