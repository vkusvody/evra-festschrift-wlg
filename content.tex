% content of the paper

% В этой статье мы расскажем про устройство глагольных полей 'искать' и 'находить' в языке кунбарланг такой-то семьи. Это язык, на котором говорят там-то и тот-то. Может быть, карта. Еще какая-нибудь социолингвистика -- про угрожаемый статус, про то, что все носители билингвы и сколько их. Может быть, яркие типологические черты.
В этой статье мы расскажем про устройство глагольных полей `искать' и `находить' (а также про некоторые смежные сюжеты) в не-пама-ньюнганском языке кунбарланг (гунвингская семья, северная Австралия). Мы показываем, что несмотря на компактность лексической системы \кун а в семантической области поиска, она отражает важные обобщения: например, разделение глаголов по ориентации на искомый объект или на место поиска, а также вторичность поля `найти' относительно `искать'. Во введении (\S\ref{sec:intro}) мы вкратце дадим сведения о социолингвистической ситуации, типологических чертах \кун а и необходимую для понимания материала грамматическую информацию. Устройство рассматриваемых лексических полей рассматривается в \S\S\ref{sec:search} и \ref{sec:find}, а в заключении (\S\ref{sec:outro}) даются некоторые общие выводы.

\section{Введение}
\label{sec:intro}
Кунбарланг генетически и структурно близок гунвингскому языку пининь кун-вок (также известному как гунвиньгу или майяли; \cite{bgw}). На сегодняшний день на кунбарланге говорят приблизительно 35 человек в трёх населённых пунктах в Арнем-ленде на крайнем севере Австралии. Все взрослые кунбарланги многоязычны; большинство свободно говорят на не-пама-ньюнганских языках маунг (семья ивайджа) или нджеббана (манингридская семья), а также в разной степени владеют английским языком. Этнические дети-кунбарланги не усваивают родной язык вследствие конкуренции с этими другими языками, и кунбарланг находится под угрозой исчезновения.

С точки зрения структуры, \кун\ является языком полисинтетического строя с такими характерными чертами как полиперсональное согласование, инкорпорация имён и наречий в глагольную словоформу, грамматически свободный порядок слов и про-дроп. Все глагольные формы \кун а финитны, и любая такая форма может быть полноценной самостоятельной клаузой. Залогов (например, страдательного) в \кун е нет, а также нет лабильности (как чёткого системного явления несводимого к про-дропу), поэтому когда мы обсуждаем глаголы и их аргументы, речь об изменении диатезы не идёт. Есть, однако, продуктивные морфологические операции, повышающие или понижающие валентность (бенефактив и рефлексив, соответственно). С точки зрения субъектов, \кун\ это язык аккузативного строя, а с точки зрения объектов --- секундативного, то есть пациентивный аргумент монотранзитивного глагола отражается в согласовании так же, как реципиент-подобный аргумент дитранзитивного глагола  
% Effectively, \rex{def:obj} amounts to a generalization that \wlg\ shows \textsc{secundative alignment} (or, more specifically, \textsc{secundative indexing}) of objects, i.e.\ the Patient-like arguments of monotransitives are treated in the same way as the Recipient/Goal/Source-like arguments of ditransitives with respect to \textbf{agreement}
\parencite{haspelmath05}. %The following pair of examples shows that the second person Theme of the transitive verb
Следующая пара примеров показывает, что пациенс глагола -\textit{burrbunj} `знать' \gfr{ex:tandr.t} %receives the same object agreement as the Goal of the ditransitive verb
занимает ту же согласовательную позицию, что реципиент глагола -\textit{wunj} `дать' \gfr{ex:tandr.r}.\footnote{В статье используются следующие сокращения: \printglossary[style=inline,type=\leipzigtype]} Поэтому в статье мы говорим не о прямых и косвенных объектах, а о первых и вторичных.
\pex<ex:tandr>\a<t>\begingl
\gla Ngayi \textbf{nga}-\textbf{ngun}-\textbf{burrbunj} Mary.//
\glb я \Fsg.\Real-\Ssg.\Obj-знать.\Np{} М.//
\glft `Я знаю тебя, Мэри.'\trailingcitation{[записи И. О'Киф 2006 г.]}//%%20060901IB02/07:07--09]}//
\endgl\a<r>\begingl
\gla Korro ngudda kun-nungku yalbi kadda-bum ninda la \textbf{rrubbiya} balkkime \textbf{kanjbadda}-\textbf{ngun}-\textbf{wunj}.//
\glb \korro{} ты \Cliv-ты.\Gen{} страна \Tpl.\Real-ударять.\Pst{} \ninda{} \la{} деньги сегодня \Tpl.\Fut-\Ssg.\Obj-дать.\Np{}//
\glft `Они убили этого [крокодила] на твоей земле и они сегодня заплатят тебе денег.'\trailingcitation{[записи И. О'Киф 2006 г.]}//%%20060620IB04/11:36--45]}//
\endgl\xe

Именная морфология слабо развита, структурный падеж сохраняется только у личных местоимений, на существительных изредка маркируются пространственные падежи. Таким образом, в принципе возможна неоднозначность в интерпретации грамматической функции произвольной именной группы; на практике она возникает редко в силу семантико-прагматических соображений. Основной согласовательной категорией в именной группе является грамматический род. В \кун е четыре грамматических рода\footnote{Если учитывать указательные местоимения, то можно выделять пять грамматических родов.}, которые определяются семантически и отражают культурную классификацию животного и растительного мира, а также предметов быта.

Кунбарланг мало описан: существует короткая тагмемная грамматика \cite{harris69} и ряд неопубликованных работ К. Колман --- дипломная работа \cite{coleman82}, рукописи разговорника \parencite{wordgra} и словаря \parencite{coleman10}. В настоящий момент в университете Мельбурна готовится полное грамматическое описание языка \parencite{ikwlg}.  Материал для настоящей статьи был собран вторым автором во время полевой работы в деревне Варруви (о.\ Южный Голбурн), кроме тех случаев, где источник указан дополнительно.

\subsection{Лексическая семантика}
\label{sec:lexsemover}
В лексической семантике \кун\ демонстрирует характерные для австралийских языков черты. В связи со сложной системой социальной организации (классификационная система родства, охватывающая весь универсум людей и в некоторой степени даже животных), термины родства --- как обращения (terms of address; функционально --- вокативы), так и референциальные термины (terms of reference) --- очень богаты и разнообразны.\footnote{Ср.: ``Reflecting its social importance, kinship is arguable the most highly elaborated semantic domain in many Australian languages'' \autocite[304]{gabysinger14}.} Так, в \кун е насчитывается порядка 25 терминов родства.

Также очень подробен и богат этнобиологический лексикон. Вследствие того, что австралийские аборигены до колонизации жили охотой и собирательством, и такой образ жизни требовал тонкого понимания живой природы, в любом австралийском языке есть большой и подробный пласт лексики, посвящённой биологическим видам. Тесная связь австралийских аборигенов с природой проявляется также и в системах классификации флоры и фауны собственно в языке (в \кун е это грамматические роды) и, кроме того, в мифологии, песнях и в таких культурных практиках как танец и изобразительное искусство.

В то же время, некоторые другие области лексики развиты непривычно слабо для носителя, например, индоевропейского языка. Так, например, в \кун е при изобилии видовых терминов часто отсутствуют родовые термины \parencite[см.\ об этом в][\S2.1, и цитируемые там источники]{gabysinger14}. Существует много названий для отдельных видов ракообразных и моллюсков, но нет слова для ракообразных вообще или для моллюсков вообще. Одним словом \textit{maworord} обозначаются лист, цветок, лепесток и травинка. Практически полностью отсутствуют кулинарные термины: есть слова \textit{neyang} `растительная еда' и \textit{kakkin} `мясо', но нет никаких различий по способу приготовления пищи; обычно просто называется вид съедаемого растения или животного.

Обращаясь к предмету настоящей статьи, мы находим, что глагольные поля \кун а представляют собой скорее последний случай. Несмотря на то, что в языке есть много глаголов с узкоспециальной семантикой, на практике чаще используются глаголы с широким значением, например -\textit{ngundje} `говорить; делать', -\textit{karrme} `держать; трогать', -\textit{marnbunj} `делать'. В известном смысле это напоминает глагольную лексику английского языка, которая так же позволяет обходиться небольшим количеством общих глаголов (хоть и с фразовыми частицами), при том что существует огромное количество глаголов с очень конкретной семантикой. В следующих двух разделах мы рассмотрим, какие лексические средства используются в \кун е для семантических полей `искать' и `находить'.

%Что касается устройства лексических полей, здесь кунбарланг проявляет типичную для австралийских языков странность: неожиданные для носителей европейских языков поля оказываются бедными и наоборот. Например, родство и всякие животные/природа -- богатые. А еда и что-то еще бедные. Глагольные поля -- это скорее последний случай. Много доминантных систем, глаголов с очень широким значением (например). Это не значит, что периферийных нет. Аналогия с английским.

\section{Искать}
\label{sec:search}
Мы идентифицировали три глаголя семантического поля `искать' в \кун е. Один из них ориентирован на искомый объект, а два других --- на место поиска.
\subsection{Глагол -\textit{yawanj}}
\label{sec:yaw}
Большинство ситуаций искания описываются в \кун е доминантным глаголом -\textit{yawanj} `\yaw'. Так ищут как одушевленный \rex{human}, так и неодушевленный \rex{nonhuman} объекты.
\ex<human>\begingl
\gla \textbf{Nga}-\textbf{yawanj} \textbf{nakarrmanj} nga-mabulunj nganjdji-wokdja.//
\glb \Fsg.\Real-\yaw.\Np{} племянник \nga-хотеть.\Np{} \Fdu.\Excl.\Fut-говорить.\Np{}//
\glft `Я ищу своего племянника, я хочу с ним поговорить.'//%gDoc:g2
\endgl \xe

\ex<nonhuman> \begingl
\gla \textbf{Nga}-\textbf{yawanj} \textbf{bi}-\textbf{ngaybu} \textbf{sunglass} nga-warre \textbf{nga}-\textbf{yawanj}.//
\glb \nga-\yaw.\Np{} \bi-я.\Gen{} sunglass \nga-\warre.\Np{} \nga-\yaw.\Np{}//
\glft `Я ищу свои солнечные очки.'//%gDoc:g3
\endgl \xe

Объект поиска может быть не только референтным, как в примерах (\getref{human}--\getref{nonhuman}), но и нереферентным. Например, с помощью этого глагола можно искать себе друзей \rex{friend} или пресную воду \rex{water}.

\ex<friend> \begingl
\gla Ninda nawalak na-kerrkung \textbf{ka}-\textbf{bun}-\textbf{yawanj} \textbf{na}-\textbf{barrkidbe} \textbf{djarrangalanj} nayi bi-rnungu friend.//
\glb \Dem.\Prox.\Cli{} ребёнок \Cli-новый \Tsg.\Real-\Tsg.\Obj-\yaw.\Np{} \Cli-другой мальчик \Nm.\Cli{} \bi-он.\Gen{} друг//
\glft `Это ребенок в школе новенький, он ищет друга.'//%gDoc:g10; в доке --- nakerrbung
\endgl \xe

\ex<water>\begingl
\gla Ki-warreni ki-wokdji \textbf{ki}-\textbf{yawani} \textbf{njunjuk} bonj\char`~bonj.//
\glb \Tsg.\irrpst-\warre.\irrpst{} \Tsg.\irrpst-говорить.\irrpst{} \Tsg.\irrpst-\yaw.\irrpst{} вода \rdp\bonj{}//
\glft `Они [журавли], бывало, летят, кричат, снова ищут воду.'\trailingcitation{[записи Анг Си в Манингриде]}//%%20150212AS02\_brolga\_transcript/01:59--02:03]}//
\endgl \xe

Морфологически на глаголе маркируются лицо и число субъекта и объекта поиска. Объект третьего лица единственного числа обычно нулевой и выражается префиксом \textit{bun}- только если его референт одушевленный, а субъект --- тоже третьего лица единственного числа \rex{friend}.

%Во всех остальных случаях (которых в наших примерах большинство) этот объект нулевой (как, например, в \rex{human}).

Здесь заслуживает упоминания одна характерная конструкция, которая хоть и не относится специфически к ситуациям искания, но тем не менее часто в них встречается. Речь идёт об имперфективной конструкции с глаголами позиции, которая выражает длительность действия во времени. Так называемые глаголы позиции, которые в ней используются, это -\textit{rna} `сидеть', -\textit{dja} `стоять', -\textit{yuwa} `лежать', а также глагол ненаправленного движения -\textit{warre} `\warre'. Примеры её в ситуации поиска --- \rex{human} и \rex{water}, где как раз используется глагол -\textit{warre} `\warre'. В \кун е не грамматикализовано противопоставление по виду, и эта конструкция предоставляет возможность подчеркнуть, что действие развёрнуто во времени (имперфектив). В настоящий момент правила распределения вспомогательных глаголов позиции мало изучены и требуют отдельного лексико-семантического исследования. Очевидны, по крайней мере, два факта: во-первых, глагол часто сохраняет рудиментарную семантику позиции/движения, но, во-вторых, она не служит строго для определения правильного глагола (то есть, может не сочетаться с типом ситуации буквально). Кажется, что наиболее часто встречается глагол -\textit{rna} `сидеть'. Пример \rex{ex:posmove} иллюстрирует перечисленные свойства: ситуация развёрнута во времени, а глагол `сидеть' не предполагает, что приходящие прибывали в сидячем положении.
\ex<ex:posmove>\begingl
\gla Kadda-rninganj kadda-nganj-kidanj.//
\glb \Tpl.\Real-сидеть.\Pst{} \Tpl.\Real-\Hith-идти.\Pst{}//
\glft `Все сходились [прибывали по-отдельности, а не одной компанией].'//%\trailingcitation{[ik160429-000/07:33--35]}//
\endgl\xe
% \a<sitgo>\begingl
% \gla {Kukka ngundje} \textbf{kadda}-\textbf{rna} \textbf{kadda}-\textbf{wonj} karra yalbi.//
% \glb maybe \Tpl.\Real-sit.\Np{} \Tpl.\Real-return.\Np{} \karra{} country//
% \glft `Maybe they're going back home.'\trailingcitation{[IK1-170615\_1SY-02/06:52--55]}//
% \endgl\xe

Возращаясь собственно к глаголу -\textit{yawanj}, отметим, что третий участник ситуации поиска --- место --- на глаголе не маркируется. Он выражается отдельной предложной группой с универсальным предлогом \textit{korro} `\korro':\footnote{Предложная группа \textit{korro kungad} `у заводи' может, по всей видимости, относиться как к `искать', так и к `пить'.}
%@ need to listen to this example...
\ex<ex:pp>\begingl
\gla Nganj-ka nganj-yawanj nayi kunj \textbf{korro} \textbf{kadda}-\textbf{dja} \textbf{kadda}-\textbf{bardi}-\textbf{djinj} \textbf{njunjuk} korro kungad.//
\glb \Fsg.\Fut-идти.\Np{} \Fsg.\Fut-\yaw.\Np{} \Nm.\Cli{} кенгуру \korro{} \Tpl.\Real-стоять.\Np{} \Tpl.\Real-жидкость-есть.\Np{} вода \korro{} заводь//
\glft `Я пойду искать кенгуру там, где они пьют воду, у заводи.'//%mm_160817
\endgl\xe

В этом отношении глагол -\textit{yawanj} в \кун е стоит в одном ряду в русским \textit{искать} и английским \textit{seek}. Во всех этих словах место поиска не входит в число аргументов глагола и может быть выражено только адъюнктом. В \кун е есть два способа передать идею поиска с акцентом на месте: -\textit{birrdjuwa} `\bdj' -\textit{rlarlakka} `\rlk'. Ниже мы рассмотрим их по-очереди.
%Согласно классификации когнитивных профилей Ван Хенке (2003), все эти глаголы относятся к типу \textsc{seek}: выделенный участник ситуации в таких словах -- желаемый объект, а область поиска занимает периферийную позицию. 

\subsection{Глагол -\textit{birrdjuwa}}
\label{sec:bdj}
Первый способ сделать акцент на месте поиска --- глагол -\textit{birrdjuwa} `\bdj' (этимологически от \textit{birr}- `рука' и -\textit{djuwa} `протыкать'). В прямом значении этот глагол описывает уборку метлой или граблями \rex{rake}. Редупликация в \rex{rake} возможна, но необязательна, она передаёт идею сравнительно большей площади уборки.
\ex<rake>\begingl
\gla \textbf{Nganj}-(\textbf{birri}\char`~)\textbf{birrdjuwa} rubbish.//
\glb \Fsg.\Fut-(\rdp)\bdj.\Np{} мусор//
\glft `Я подмету/уберу граблями мусор.'//%gDoc:g7
\endgl\xe

Образ очищения поверхности/фона от загораживающих или загромождающих элементов, каких-то помех и т.п.\ послужил основой ряда метафорических переносов. Например, этим глаголом небо может расчиститься от облаков \rex{sky}, им же можно открыть дверь (буквально, очистить проход от двери, \rex{door}).
\ex<sky>\begingl
\gla Kuyunu \textbf{ka}-\textbf{birrdjuwa}.//
\glb облако \Tsg.\Real-\bdj.\Np{}//
\glft `Облака расчищаются.'\trailingcitation{[\cite[16]{coleman10}]}//
\endgl\xe

\ex<door>\begingl
\gla Djarderre$=$rnungu \textbf{ka}-\textbf{birrdjuwa}.//
\glb рот$=$он.\Gen{} \Tsg.\Real-\bdj.\Np{}//
\glft `Он открывает дверь [букв.\ `расчищает вход'].'\trailingcitation{[\cite[17]{coleman10}]}//
\endgl\xe

Как видно из сопоставления примеров \rex{sky} и \rex{door}, синтаксическим объектом глагола может быть как загораживающий элемент, так и очищаемое от него пространство.

Расчищать пространство можно не только с целью сделать его чистым, но и для того чтобы что-нибудь найти. Мы предполагаем, что здесь и возникает идея поиска. %Так, расчищать граблями землю можно с целью найти что-либо под ворохом листьев или веток \rex{searchground}. А
Наши данные по использованию этого глагола в семантическом поле поиска в настоящий момент ограничены одной типовой ситуацией: поиск чего-то в книге (например, нужного рассказа в сборнике; \rex{searchbook}). Мы анализируем это употребление через ту же идею поиска как процесса \textit{открытия} (ср.\ английский глагол \textit{discover} `найти', передающий похожую идею). Так, перелистывая страницы ищущий как бы очищает новые страницы от загораживающих их предыдущих. %перелистывать страницы книги можно для того, чтобы найти в книге какую-то конкретную информацию \rex{searchbook}.
В таких употреблениях -\textit{birrdjuwa} сближается с русскими глаголами \textit{обыскивать} и \textit{рыться}, у которых первый объект это место поиска.
% \ex<searchground>\begingl
% \gla example with noun incorporation -- from dict?.//
% \glb //
% \glft `transl.'\trailingcitation{[src]}//
% \endgl\xe
%@ need to relisten this one too --- what the hell is going on with the tenses???
\ex<searchbook>\begingl
\gla \textbf{Nganj}-\textbf{birri}\char`~\textbf{birrdjuwa} \textbf{djurra} nga-yawanj story mankurdel.//
\glb \Fsg.\Fut-\rdp\bdj.\Np{} бумага \nga-\yaw.\Np{} история великан//
\glft `Я полистаю книгу в поисках истории про великана-людоеда.'//%gDoc:g7
\endgl\xe

Как было сказано выше, первый объект глагола -\textit{birrdjuwa} это либо место поиска (что именно расчищается), либо помеха, от которой это место освобождается. В ситуации с книгой эти две семантических роли неразличимы. Во всех примерах выше этот аргумент выражается независимой именной группой, но также возможно дублирование его в виде инкорпорированного имени в глаголе \rex{ex:yalbi}. %@ стоянка, лагерь, страна...
\ex<ex:yalbi>\begingl
\gla Yalbi ka-\textbf{ngundek}-birrdjuwa.//
\glb страна \Tsg.\Real-страна-\bdj.\Np{}//
\glft `Она убирается на стоянке/в лагере.'\trailingcitation{[\cite[17]{coleman10}]}//
\endgl\xe

Объект поиска при этом предикате выразить нельзя (как и у русских глаголов \textit{рыться} и \textit{обыскивать}). Для того, чтобы его назвать, необходимо употребить доминантную лексему поля -\textit{yawanj} \rex{searchbook}.

Нетривиальная семантика глагола -\textit{birrdjuwa} в прямых употреблениях сильно ограничивает круг допустимых ситуаций поиска. Для того, чтобы употребление глагола было возможно, ситуация должна предполагать физическое очищение пространства от ненужных объектов. %Так, искать что-то в груде мусора, в земле или в книге при помощи этого глагола можно а,
Таким образом, наш анализ предсказывает, что искать иголку в стоге сена следует именно при помощи этого глагола. Напротив, искать ракушку на дне, ныряя, должно быть невозможно (если только она не схоронена под слоем песка). Эти предсказания ещё предстоит проверить в ходе будущей работы. В то же время мы установили, что перебор страниц не является определяющим компонентом при поиске в книге: например, искать себе друзей (как бы перебирая людей или очищая общество от `ненужных' людей в поиске друга) таким образом нельзя. Когда носителю предъявили для оценки такой пример, составленный лингвистом \rex{children}, %(`Он-\textit{birrdjuwa} детей') и спросили, что оно могло бы значить,
он проинтерпретировал его следующим образом: ``Он стягивает с детей одеяла''.
\ex<children>\begingl
\gla Ka-buddu-birrdjuwa barrayidjyidj.//
\glb \Tsg.\Real-\Tpl.\Obj-\bdj.\Np{} children//
\glft `Он стягивает с детей одеяла.'//%gDoc:j10
\endgl\xe

Это означает, что идея физического открытия некоего объекта (в данном случае, детей) важнее идеи последовательного перебора (которая возникает в случае с книгой просто как артефакт устройства книги-кодекса).
\subsection{Глагол -\textit{rlarlakka}}
\label{sec:rlakka}
Другой способ выразить идею поиска с акцентом на месте --- глагол -\textit{rlarlakka} `\rlk', который представляет собой редуплицированную форму глагола -\textit{rlakka} `бросать' \rex{throw}.
\ex<throw>\begingl
\gla Ka-rlakwang karlikarli ka-nganj-wom ka-bun-bum.//
\glb \Tsg.\Real-бросать.\Pst{} бумеранг \Tsg.\Real-\Hith-возвращаться.\Pst{} \Tsg.\Real-\Tsg.\Obj-ударять.\Pst{}//
\glft `Он бросил бумеранг, а тот вернулся и ударил его.'//%nancy_160524
\endgl\xe

Если же употребить этот глагол в редуплицированной форме -\textit{rla\char`~rlakka} с существительным `сумка' в качестве первого объекта, фраза приобрает значение `рыться в сумке' \rex{bag}. Редупликация здесь передает идею множественности действия (`разбрасывать').
\ex<bag>\begingl
\gla Nga-warrenj nga-yawang ngarnilmaddjiyi \textbf{nga}-\textbf{rlarlakwang} baladdji babi la nga-rnay kuwalak, ngarnilmaddjiyi nga-yawang karlu.//
\glb \nga-\warre.\Pst{} \nga-\yaw.\Pst{} очки \nga-\rlk.\Pst{} сумка потом \la{} \nga-видеть.\Pst{} деньги очки \nga-\yaw.\Pst{} \karlu{}//
\glft `Я перерыл сумку в поисках очков, но нашел только немного денег.'//%gDoc:h18
\endgl\xe

Так же, как и лексема -\textit{birrdjuwa}, -\textit{rlarlakka} допускает дублирование первого объекта в виде инкорпорированного корня \rex{baginc}. С этим глаголом тоже нельзя выразить искомый объект --- ни аргументом, ни адъюнктом --- для этого употребляется доминантное слово -\textit{yawanj}.
\ex<baginc>\begingl
\gla Ki-ngaybu burruburrukang djamun kadda-yambi-rlarlakka kadda-yawanj kun-bareng la maworord.//
\glb \ki-я.\Gen{} сумка полицейский \Tpl.\Real-скарб-\rlk.\Np{} \Tpl.\Real-\yaw.\Np{} \Cliv-опасный \la{} трава//
\glft `Полицейские обыскивают мою сумку в поисках травы и алкоголя.'//%gDoc:g6
\endgl\xe

%\ex<ex:wfound>\begingl
%\gla Nga-warrenj nga-yawang la babi la \textbf{nga}-\textbf{warrenj} \textbf{la} \textbf{nga}-\textbf{rnay}.//
%\glb \Fsg.\Real-\warre.\Pst{} \Fsg.\Real-искать.\Pst{} \la{} позже \la{} \Fsg.\Real-\warre.\Pst{} \la{} \Fsg.\Real-видеть.\Pst{}//
%\glft `Я искал нечто, а потом (\textbf{я шарился и}) \textbf{нашёл}.'\trailingcitation{[IK1-170610\_1SY-02/54:02--10]}//
%\endgl\xe

\section{Находить}
\label{sec:find}

В лексической зоне `находить' \кун\ демонстрирует яркий пример ассиметрии антонимичных семантических полей. Если `искать' покрывает один доминантный
глагол поля, то в зоне `находить', строго говоря, слов нет вообще. Большинство ситуаций нахождения объекта передаются в этом языке при помощи предиката
`смотреть/видеть' -\textit{rnanj}. Сама связь смотрения и обнаружения кажется типологически частотной. Даже на ограниченной выборке языков, представленной в настоящем сборнике, она встретилась несколько раз в неродственных языках. Так, например, в нганасанском смысл 'найти' выражается корнем ŋətə-, который в производнных основах значит `видеть/увидеть': \textit{ŋəδüˀ}- `видеть', \textit{ŋətu}- (статив) `быть видным' и \textit{ŋətu}-\textit{m}- `появиться' (Гусев, настоящий сборник). Особенность кунбарланга состоит в том, что два смысла `видеть' и `находить' в нем выражаются не разными однокоренными формами, а одним и тем же глаголом. Так, в примере \rex{see} -\textit{rnanj} передает значение `видеть', а в примере \rex{money} --- `находить'.

\ex<money>\begingl
\gla example.//
\glb I-went PREP road I-saw money it-was.lying//
\glft `translation'//
\endgl\xe

Спектр ситуаций нахождения, которые можно передать с помощью этого глагола, довольно широк. Им можно найти вещь как случайно \rex{money}, так и в результате поиска \rex{sunglasses}:

\ex<sunglasses>\begingl
\gla example.//
\glb he-existed he-sought sungl. then and he-saw//
\glft `translation'//
\endgl\xe

Найти потерянный объект предикатом -\textit{rnanj} можно не только для себя, но и для кого-то другого \rex{bookforme}. В таком случае на глаголе дополнительно маркируется лицо и число бенефицианта (в данном примере, -\textit{...}-). Согласование по бенефицианту при его наличии является типичным для кунбарлангских глаголов. 

\ex<bookforme>\begingl
\gla example.//
\glb he-me-BEN-saw the book//
\glft `translation'//
\endgl\xe

Наконец, этим глаголом можно обнаружить (или не обнаружить, как в примере ниже) объект или человека --- то есть найти его, но не присвоить в себе. В русском языке в таких ситуациях, помимо \textit{обнаружить}, употребляются предикаты \textit{встретить} и \textit{увидеть}.

\ex<notseeanyone>\begingl
\gla example.//
\glb  I-returned I-flipped not who I-saw.NEG//
\glft `translation'//
\endgl\xe

Интересная ассиметрия в употреблении этого глагола возникает в контексте отрицания. Так, если для выражения утвердительного смысла употребление -\textit{rnanj} необходимо, то для передачи значения `не нашел' --- нет. Например, если человек искал вещь, но не смог её найти, более естественно передать эту идею
с помощью предикативного опущения: ``я искал --- [этого] нет'':

\ex<nobook>\begingl
\gla example.//
\glb I I-existed I-sought no the book and he he-saw for-me//
\glft `translation'//
\endgl\xe

А если человек ничего специального не искал, но посмотрел вокруг себя и чего-то не увидел (т.е. не обнаружил вещи), то помимо обычного "не нашел" (\rex{notseeanyone}) возможна и частотна конструкция "я посмотрел --- нет". И так как глагол -\textit{rnanj} в одном из своих значений передает смысл 'смотреть/видеть', то возникает, на первый взгляд, противоречивая ситуация: смысл `не нашел' передается при помощи -\textit{rnanj} в утвердительной форме:

\ex<lookno>\begingl
\gla example.//
\glb I-returned {I-looked / *I-flipped} they-went//
\glft `translation'//
\endgl\xe



ср.\ \rex{bag} выше
\ex[exno={\getref{bag}$'$}]<ex:notfound>\begingl
\gla Nga-warrenj nga-yawang ngarnilmaddjiyi nga-rlarlakwang baladdji babi la \textbf{nga}-\textbf{rnay} \textbf{kuwalak}, ngarnilmaddjiyi nga-yawang karlu (\textbf{ngurnda} \textbf{ngay}-\textbf{rnani}).//
\glb \nga-\warre.\Pst{} \nga-\yaw.\Pst{} очки \nga-\rlk.\Pst{} сумка потом \la{} \nga-видеть.\Pst{} деньги очки \nga-\yaw.\Pst{} \karlu{} \phantom{(}\Neg{} \Fsg.\irrpst-видеть.\irrpst{}//
\glft `Я перерыл сумку в поисках очков, но нашел только немного денег, а очков не нашёл.'//%gDoc:h18
\endgl\xe

\ex<ex:foundben>\begingl
\gla Ka-\textbf{ngan}-\textbf{marnanj}-rnay nayi djurra.//
\glb \Tsg.\Real-\Fsg.\Obj-\Ben-видеть.\Pst{} \Nm.\Cli{} бумага//
\glft `Он мне нашёл книжку.'//%gDoc:g21
\endgl\xe


\section{Заключение}
\label{sec:outro}

В зонах `искать' и `находить' кунбарланг демонстрирует типичную структуру глагольных полей для австралийского языка: наличие одного многозначного доминантного глагола и, опционально, нескольких узкоспециальных. Анализ этих зон позволил выявить в кунбарланге ряд типологически интересных черт. Одной из них является лексическое противопоставление объекта и места поиска в поле `искать'. Так -\textit{yawanj} может иметь в качестве аргумента только объект, а -\textit{birrdjuwa} и -\textit{rlarlakka} --- только место. Другой вклад в типологию этих зон --- источники для рассматриваемых значений. 



%Источники: связь искания с очищением, а нахождения со смотрением
% Наконец, в кунбарланге проявляется наглядно ассиметрия двух полей и вторичных статус идеи 'найти' по отношению  к идее 'искать'. Несмотря на то, что эту вторичность можно обнаружить уже при анализе утвердительных контекстов (наличие специализированного глагола искать и отсутствие его для найти), намного более четко эта разница видна в отрицательных контекстах. Иллюстративность их состоит не только в отсутствии глагола, но и в выборе разных описательных конструкций для разных ситуаций не нахождения. Так, не найти в результате поиска передается одним способом, а не найти при отсутствии предварительного поиска - другим. Эта оппозиция, типологически подтверждающаяся, не видна в утвердительных преложениях. Получается необычная комбинация: мы видим наглядно оппозицию поля в контекстах, в которых ни одного глагола поля не употреблено. Более глобально, эти данные показывают, что для увеличения потенциала лексической типологии может быть не лишним иногда заглянуть в зону синтаксиса и в случае с предикатами отдельно анализировать предлоежения с отрицанием.

%%% Local Variables:
%%% mode: latex
%%% TeX-master: "main"
%%% End:
