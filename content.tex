% content of the paper
\section{Введение}
Всем читать \cite{ik16bun}!
% В этой статье мы расскажем про устройство глагольных полей 'искать' и 'находить' в языке кунбарланг такой-то семьи. Это язык, на котором говорят там-то и тот-то. Может быть, карта. Еще какая-нибудь социолингвистика -- про угрожаемый статус, про то, что все носители билингвы и сколько их. Может быть, яркие типологические черты.

%Что касается устройства лексических полей, здесь кунбарланг проявляет типичную для австралийских языков странность: неожиданные для носителей европейских языков поля оказываются бедными и наоборот. Например, родство и всякие животные/природа -- богатые. А еда и что-то еще бедные. Глагольные поля -- это скорее последний случай. Много доминантных систем, глаголов с очень широким значением (например). Это не значит, что периферийных нет. Аналогия с английским.

\section{Искать}
% Поле 'искать' покрывается доминантным глаголом 

\ex<ex:wfound>\begingl
\gla Nga-warrenj nga-yawang la babi la \textbf{nga}-\textbf{warrenj} \textbf{la} \textbf{nga}-\textbf{rnay}.//
\glb \Fsg.\Real-\warre.\Pst{} \Fsg.\Real-искать.\Pst{} \la{} позже \la{} \Fsg.\Real-\warre.\Pst{} \la{} \Fsg.\Real-видеть.\Pst{}//
\glft `Я искал нечто, а потом (\textbf{я шарился и}) \textbf{ нашёл}.'\trailingcitation{[IK1-170610\_1SY-02/54:02--10]}//
\endgl\xe

\section{Находить}
% А поле \textsc{находить} покрывается ещё более доминантным глаголом.

\section{Заключение}